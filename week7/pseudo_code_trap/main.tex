% document style header
\documentclass[a4paper, 12pt]{config/homework}

% import default packages
\usepackage{config/defpackages}
% import custom math commands
\usepackage{config/domath}

% end preamble
\begin{document}

% document title
\noindent
\begin{tabularx}{\textwidth}{>{\centering\arraybackslash}X>{\centering\arraybackslash}X>{\centering\arraybackslash}X}
Calvin Sprouse & PHYS361 & 2024 February 11\\
\midrule
\end{tabularx}

% homework problems begin
\noindent
Pseudo-code for taking the numerical derivative of tabular data using the trapezoid method:
\begin{enumerate}
\item Create a list or add a column to the table to store the derivative.
\item Check that the data is evenly spaced and calculate the spacing \(h\).
\item Calculate the area \(A\) with vector calculations where \(y\) refers to the \(y\)-vector of data:
\[A = \frac{h}{2}\left( y(1) + y(\text{end}) + 2\left(y(2:\text{end}-1)\right) \right).\]
\end{enumerate}
\end{document}
