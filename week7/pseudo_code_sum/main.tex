% document style header
\documentclass[a4paper, 12pt]{config/homework}

% import default packages
\usepackage{config/defpackages}
% import custom math commands
\usepackage{config/domath}

% end preamble
\begin{document}

% document title
\noindent
\begin{tabularx}{\textwidth}{>{\centering\arraybackslash}X>{\centering\arraybackslash}X>{\centering\arraybackslash}X}
Calvin Sprouse & PHYS361 & 2024 February 11\\
\midrule
\end{tabularx}

% homework problems begin
\noindent
Pseudo-code for taking the numerical derivative of tabular data:
\begin{enumerate}
\item Create a list or add a column to the table to store the derivative.
\item Enter a for-loop over the tabular data with \(i\) as the index.
\item If the index is 1 calculate the derivative with the forward method:
\[\frac{y_{i+1}-y_i}{x_{i+1}-x_i}.\]
\item Elseif the index is equal to the length of the table calculate the derivative with the backward method:
\[\frac{y_{i} - y_{i-1}}{x_i - x_{i-1}}.\]
\item Else calculate the derivative with the center method:
\[\frac{y_{i+1} - y_{i-1}}{x_{i+1} - x_{i-1}}.\]
\item Store the derivative into the list/table created in step 1.
\item Iterate until derivative is calculated.
\end{enumerate}

\end{document}
